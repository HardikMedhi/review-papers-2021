Gravitational waves was first proposed by Albert Einstein in his 1916 paper on "The Theory of General Relativity". Gravitational waves are disruptions or “ripples” of the space time fabric, caused due to the acceleration of massive bodies, such as neutron stars or black holes. These disruptions stretch and squeeze the space as they pass by, and travel at the speed of light from its source in all directions. Before the discovery of gravitational waves, we were limited to looking at the universe only using electromagnetic waves emitted by various celestial objects. We go through one of the results of general relativity which is gravitational waves and explain it's theoretical aspect like linearized theory and experimental setup that provides us with the data that we can use to understand the nature of the sources of these waves and the universe itself. Here we are reviewing some of their work to better understand not only the reasons for their use, but also the very methods in which they are formed and how we can detect them. The paper consist of LIGO, the most prominent Gravitational Wave detector made by mankind yet. We study the various disturbances affecting LIGO, and the causes behind them. The different methods used behind the extraction of signals once the noise has been reduced has been discussed as well. Finally we explain the various observations of LIGO and Advancements of LIGO. 
